\documentclass[dvips]{article}
\usepackage{graphicx}
 
\begin{document}

\title{Analysis of Gaussian Elimination with OpenMP}
 
\author{Dylan Bonsell}
 
\maketitle % this produces the title block
 
\section{Introduction}

Gaussian Elimination is a major tool in solving systems of equations. By making such a thing parallel, we can increase the time it takes to get solutions to massive problems.

\section{Analysis}
 
\subsection{Abstract}

For this assignment, I programmed a solution to the problem with two programs, row.c and col.c.

All of my times are on an i7 @ 4.8GHz x 8, 8GB of DDR3 2333 ram, and a ssd.

I ran each of the tests 5 times, averaged the time, and then did my comparisons. This way, the times are an accurate representation of the algorithm itself,
rather than having an outlier because of background CPU usage. 

\subsection{Answers}

The solutions to the book questions are:
\\*
A: No. The outer loop of the row-oriented algorithm cannot be parallelized, as each row requires the row
below it.
\\*
B: Yes. The inner loop does not overwrite itself, if compiled correctly!
To work around x[row] from overwriting, you can use the call:
 \begin{verbatim}
 #pragma omp parallel for reduction(- : local_x)
 \end{verbatim}
which will reduce, then you can put the local\_x back into x[row]
\\*
C: No. Again, the outer loop requires the previous column to have been completed.
\\*
D: Yes. There are no special calls for this one, you can simply use parallel for to make it parallel.
\\*

\subsection{Compilation \& Running}
 This program uses a standard make call to compile both.

 Both programs take in num threads from the argument list, such as:
 \begin{verbatim}
  ./Program $num_threads$
  \end{verbatim}

 If this is not included, OpenMP will determine this as per default.

\subsection{Data}
Below is the data for the experiment:
\\*
\\*
For n = 10,000, no scheduling
\\*
\begin{tabular}{ l | c r }
   & Row & Col \\
  \hline
  Average & .748 &  1.314\\
  \hline

\end{tabular}
\\*
\\*
For n = 10,000, Scheduling
\\*
\begin{tabular}{ l | c r }
   & Row & Col \\
  \hline
  Average for Guided & .882 &  1.954\\
  \hline
   Average for Static & .56 &  1.324\\
  \hline
   Average for Dynamic & 8.078 &  9.416\\
  \hline
\end{tabular}
\\*
\subsection{Data Analysis}

I found it odd that dynamic was so inefficient, as it had a magnatude of 9 times the other times. 
As for the others, it seems that static scheduling workes well, I suspected that it is the default, but was
unable to find any information on what it uses as default.

I have started working on the full gaussian elimination program, which I may finish next week after finals.

\end{document}

